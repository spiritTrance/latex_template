\documentclass{ctexart}
% 此处引入常用包,从此行到46行均无需修改
\usepackage[dvipsnames, svgnames, x11names]{xcolor}
\usepackage{listings}
\usepackage{graphicx}
\usepackage{tabularx}
\usepackage[most]{tcolorbox}
\usepackage{amsmath}
\usepackage{multicol}
\usepackage{pifont}
\usepackage{enumitem}
\usepackage{bbding}
\usepackage{colortbl}
\usepackage{placeins}
\usepackage{mathpazo}
\usepackage{bm}
\usepackage{tikz}
\usepackage{xparse}

\pagestyle{empty}


%定义题目计数器和命令
\newcounter{questioncnt}
\newcounter{subquestioncnt}[questioncnt]
\newcounter{subsubquestioncnt}[subquestioncnt]

\NewDocumentCommand\question{om}{\noindent\IfNoValueTF{#1}{\stepcounter{questioncnt}\arabic{questioncnt}\quad#2}{#1\quad#2}\par}
\NewDocumentCommand\subquestion{om}{\noindent\IfNoValueTF{#1}{\stepcounter{subquestioncnt}\arabic{questioncnt}.\arabic{subquestioncnt}\quad#2}{#1\quad#2}\par}
\NewDocumentCommand\subsubquestion{om}{\noindent\IfNoValueTF{#1}{\stepcounter{subsubquestioncnt}\arabic{questioncnt}.\arabic{subquestioncnt}.\arabic{subsubquestioncnt}\quad#2}{#1\quad#2}\par}

%定义回答计数器和命令
\newcounter{answercnt}
\newcounter{subanswercnt}[answercnt]
\newcounter{subsubanswercnt}[subanswercnt]

\NewDocumentCommand\answer{o}{\noindent\IfNoValueTF{#1}{\stepcounter{answercnt}\arabic{answercnt}}{#1}\quad}
\NewDocumentCommand\subanswer{o}{\noindent\IfNoValueTF{#1}{\stepcounter{subanswercnt}\arabic{answercnt}.\arabic{subanswercnt}}{#1}\quad}
\NewDocumentCommand\subsubanswer{o}{\noindent\IfNoValueTF{#1}{\stepcounter{subsubanswercnt}\arabic{answercnt}.\arabic{subanswercnt}.\arabic{subsubanswercnt}}{#1}\quad}

%在此处进行基本信息修改
\newcommand{\sCourse}{示例课程}   %课程名
\newcommand{\nTime}{1}             %作业次数
\newcommand{\sName}{spirittrance}           %学生姓名
\newcommand{\sNumber}{20204205}     %学号

%页边距设置
\usepackage[left=2cm,right=2cm,top=3cm,bottom=1cm]{geometry}

\begin{document}
    \setcounter{answercnt}{0}
    %标题部分修改
    \begin{center}
        \fontsize{16pt}{0}{\textbf{\kaishu\sCourse课程\quad第\nTime次作业}}\\
        \fontsize{13pt}{0}{\textit{\kaishu\sName\qquad\sNumber}}
    \end{center}

% 本模板提供了两种模式:抄录题目和不抄录题目的,可以自定义题号或者自己编号
% 第一个是带抄录题目的:用法如下
\question{\textcolor{red}{看注释进行相应修改,包括页边距和顶部的基本信息}}
\subquestion{1+1}
\subsubquestion{1+1=?}
等于3

% 【注意回答完一个问题后,空一行再写新问题】
\subsubquestion{1+2=?}
等于4

%第二个模式是只带题号的,不抄写题目,用法如下
\answer 第一题这样回答

\subanswer 第一题第一小问这样回答

\subsubanswer 第一题第一小问第一小问这样回答

%下面是自定义题号示例,注意题号是方括号
\question[A]{第一个问题}
\subquestion[A.A]{第一个问题第一小问}
\subsubquestion[A.A.A]{第一个问题第一小问第一小问}
回答1号

%记住还是要空一行
\subsubquestion[A.A.B]{第一个问题第一小问第一小问}
回答2号

\answer[1] 第一个回答

\subanswer[(1)] 第一个回答的子回答

%记住要空一行,不包括注释
\subsubanswer[A] 第一个回答的子回答的子回答
\end{document}