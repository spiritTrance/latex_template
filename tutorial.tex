\documentclass{ctexart}

\usepackage[dvipsnames, svgnames, x11names]{xcolor}
\usepackage[top=2cm,bottom=2cm,left=1cm,right=4cm]{geometry}
\usepackage[bookmarks=true,bookmarksopen=true,bookmarksnumbered=true]{hyperref}
\usepackage{listings}
\usepackage{graphicx}
\usepackage{tabularx}
\usepackage[most]{tcolorbox}
\usepackage{amsmath}
\usepackage{multicol}
\usepackage{pifont}
\usepackage{enumitem}
\usepackage{bbding}
\usepackage{colortbl}
\usepackage{placeins}
\usepackage[morefloats = 18]{morefloats}
\usepackage{mathpazo}
\usepackage{bm}
\usepackage{tikz}
\lstset{
    numbers = none ,                                    %可选参数有none,right,left
    breaklines ,                                        %换行有影响,不加这个则换行时从头开始
    numberstyle = \tiny ,                               %数字大小’
    keywordstyle = \color{blue!70} ,                    %关键字颜色
    commentstyle =\color{black!40!white} ,              %注释颜色
    frame = shadowbox ,                                 %阴影设置
    rulesepcolor = \color{red!20!green!20!blue!20} ,    %阴影颜色设置
    escapeinside =`',                                   %lst中文支持不太好,可以用这个括在中文旁边
    basicstyle =\footnotesize\ttfamily                  %代码字体设置
}
%NOTE THAT \usepackage{enumitem} \usepackage{bbding} is required
%basic defination
\newcommand{\introductionDefalutName}{内容提要}    %default title
\newcommand{\citshapeIntro}{\kaishu}            %font
\newcommand{\introDot}{\upshape\scriptsize\SquareShadowBottomRight}     %dots before items in introductions
\definecolor{structurecolor}{HTML}{0072E3} %box color
%settings of introduction environment
\tcbset{
    introductionsty/.style={
    enhanced,
    breakable,
    colback=structurecolor!10,
    colframe=structurecolor,
    fonttitle=\bfseries,
    colbacktitle=structurecolor,
    fontupper=\citshapeIntro,
    attach boxed title to top center={yshift=-3mm,yshifttext=-1mm},
    boxrule=0pt,
    toprule=0.5pt,
    bottomrule=0.5pt,
    top=8pt,
    before skip=8pt,
    sharp corners
    },
}

\newenvironment{introduction}[1][\introductionDefalutName]{
    \begin{tcolorbox}[introductionsty,title={#1}]
        \begin{multicols}{2}
            \begin{itemize}[label=\textcolor{structurecolor}{\introDot}]
}
{
            \end{itemize}
        \end{multicols}
    \end{tcolorbox}
}
%自定义盒子环境
%summarybox[color]{title}
\newenvironment{mybox}[2][\coldefault]{
    \begin{tcolorbox}[
        title=#2,
        colframe=#1,
        colback=#1!10,
        breakable
    ]
}{\end{tcolorbox}}
% \input{./mytemplate/components/myColorPlate.tex}  %provide colors, you can use the VsCode extension color highlight. e.g.#02a29f
%mydef{text}
\newcommand{\mydef}[1]{\textbf{\textcolor{blue!75!black}{#1}}}
\newcommand{\myemph}[1]{\textbf{\textcolor{red!85!black}{#1}}}

%myimage[figureScale]{path}{caption}{label}
\newcommand{\myimage}[4][1]{{
    \begin{figure}[htbp]
        \centering
        \label{fig:#4}
        \includegraphics[scale=#1]{#2}
        \caption{#3}
    \end{figure}}}
%basic defination
\newcommand{\citshapeRef}{\kaishu}            %font
\definecolor{colReference}{HTML}{FFA042}   %box color
%settings of introduction environment
\tcbset{
    referencesty/.style={
    enhanced,
    breakable,
    colback=colReference!10,
    colframe=colReference,
    fonttitle=\bfseries,
    colbacktitle=colReference,
    fontupper=\citshapeRef,
    leftrule=5pt,
    rightrule=0pt,
    toprule=0pt,
    bottomrule=0pt,
    top=8pt,
    before skip=8pt,
    sharp corners
    },
}
\newenvironment{myReference}{
    \begin{tcolorbox}[referencesty]
}
{
    \end{tcolorbox}
}
%自定义提示颜色
\definecolor{mycolwarning}{HTML}{CE0000}
\definecolor{mycolsummary}{HTML}{8CEA00}
\definecolor{mycoldefault}{HTML}{2828FF}
\definecolor{mycolquestion}{HTML}{00E3E3}
\definecolor{mycolexample}{HTML}{FFDC35}
\definecolor{mycolextend}{HTML}{FFA042}
% \newcommand{\mycolwarning}{__mycolwarning}
% \newcommand{\mycolsummary}{__mycolsummary}
% \newcommand{\mycoldefault}{__mycoldefault}
% \newcommand{\mycolquestion}{__mycolquestion}
% \newcommand{\mycolexample}{__mycolexample}
% \newcommand{\mycolextend}{__mycolextend}

% DENOTE THAT pifont package is required for command \ping

\definecolor{colRemark}{HTML}{73BF00}
\tcbset{
    remarksty/.style={
        colframe=magenta,
        colback=magenta!12!white,
        boxed title style={colback=magenta},
	    breakable,
	    enhanced,
	    sharp corners,
	    boxsep=1pt,
	    attach boxed title to top left={yshift=-\tcboxedtitleheight,  yshifttext=-.75\baselineskip},
	    boxed title style={boxsep=1pt,sharp corners},
        fonttitle=\bfseries\sffamily,
        drop lifted shadow
    },
}

\newtcolorbox{remark}[1][]{
    remarksty,
    title={\scalebox{1.75}{\raisebox{-.25ex}{\ding{43}}}~笔记},
    colframe=yellow!45!white,
    colback=yellow!45!white,
    coltitle=colRemark,
    fontupper=\sffamily,
    boxed title style={colback=yellow!45!white},
    boxed title style={boxsep=1ex,sharp corners},%%
    overlay unbroken and first={
        \node[below right,font=\normalsize,color=red,text width=.8\linewidth]
        at (title.north east) {#1};
    }
}
%Theorem environment
    %Here to define a counter
\newcounter{myTheoremCounter}[subsection]
    %Here to define a auto counter
\newcommand{\myAutoTheoremCounter}{\stepcounter{myTheoremCounter}\thesubsection.\themyTheoremCounter}
\definecolor{myTheoremColor}{HTML}{FFDC35}
\tcbset{
    myTheoremStyle/.style={
    enhanced,
    breakable,
    colback=myTheoremColor!10,
    colframe=myTheoremColor,
    fonttitle=\bfseries,
    colbacktitle=myTheoremColor,
    fontupper=\citshapeIntro,
    attach boxed title to top left={},
    boxed title style = {sharp corners},
    boxrule=1pt,
    toprule=1pt,
    bottomrule=1pt,
    top=8pt,
    before skip=8pt,
    sharp corners,
    },
}
%Title needed
\newenvironment{myTheorem}[1]{
    \begin{tcolorbox}[
        myTheoremStyle,
        title=\color{black}{定理\myAutoTheoremCounter\quad #1}
    ]
        
}{\end{tcolorbox}}
%Defination environment
    %Here to define a counter
    \newcounter{myDefinationCounter}[subsection]
    %Here to define a auto counter
\newcommand{\myAutoDefinationCounter}{\stepcounter{myDefinationCounter}\thesubsection.\themyDefinationCounter}
\definecolor{myDefinationColor}{HTML}{01814A}
\tcbset{
    myDefinationStyle/.style={
    enhanced,
    breakable,
    colback=myDefinationColor!10,
    colframe=myDefinationColor,
    fonttitle=\bfseries,
    colbacktitle=myDefinationColor,
    fontupper=\citshapeIntro,
    attach boxed title to top left={},
    boxed title style = {sharp corners},
    boxrule=1pt,
    toprule=1pt,
    bottomrule=1pt,
    top=8pt,
    before skip=8pt,
    sharp corners
    },
}
%Title needed
\newenvironment{myDefination}[1]{
    \begin{tcolorbox}[
        myDefinationStyle,
        title=\color{white}{定义\myAutoDefinationCounter\quad #1}
    ]
        
}{\end{tcolorbox}}
\newcounter{myPropertyCounter}
    %Here to define a auto counter
\newcommand{\myAutoPropertyCounter}{\stepcounter{myPropertyCounter}\themyPropertyCounter}
\definecolor{myPropertyColor}{HTML}{46A3FF}
\newenvironment{myProperties}{
    \setcounter{myPropertyCounter}{0}
    \begin{itemize}[label=\textbf{\color{myPropertyColor}{性质\myAutoPropertyCounter}}]

}{\end{itemize}}
\definecolor{__selWordColor__}{HTML}{EA7500} % default: 0,124,53
\definecolor{__ledLineColor__}{HTML}{EA7500} % default: 153,255,153
\definecolor{__sideBoxColor__}{HTML}{ACD6FF} % default: 216,255,216

\newcommand{\elegantpar}[2]{%
  \textcolor{__selWordColor__}{$\bm\langle{}\!{}$#1${}\!{}\bm\rangle$}%
  \begin{tikzpicture}[remember picture, baseline=-0.75ex]%
    \node[coordinate] (inText) {};%
  \end{tikzpicture}%
  \marginpar{%
    \renewcommand{\baselinestretch}{1.0}%
    \begin{tikzpicture}[remember picture]%
      \draw node[fill= __sideBoxColor__, rounded corners,text width=\marginparwidth] (inNote){\footnotesize#2};%
  \end{tikzpicture}%
  }%
  \begin{tikzpicture}[remember picture, overlay]%
    \draw[draw = __ledLineColor__, thick]
      ([yshift=-0.55em] inText)
        -| ([xshift=-0.55em] inNote.west)
        -| (inNote.west);%
  \end{tikzpicture}%
}

\setlength{\marginparwidth}{2.5cm}

\begin{document}
    \tableofcontents
    \newpage

    \section{前言}
    部分模板参考overleaf上的项目(\mydef{bbe Book Template}和\mydef{ElegantBook Template}),这些模板都很不错www,推荐去体验一下。

    \section{如何使用文档}
    在导言区引入下面语句进行使用:
    \begin{lstlisting}
        \usepackage[dvipsnames, svgnames, x11names]{xcolor}
\usepackage[top=2cm,bottom=2cm,left=1cm,right=4cm]{geometry}
\usepackage[bookmarks=true,bookmarksopen=true,bookmarksnumbered=true]{hyperref}
\usepackage{listings}
\usepackage{graphicx}
\usepackage{tabularx}
\usepackage[most]{tcolorbox}
\usepackage{amsmath}
\usepackage{multicol}
\usepackage{pifont}
\usepackage{enumitem}
\usepackage{bbding}
\usepackage{colortbl}
\usepackage{placeins}
\usepackage[morefloats = 18]{morefloats}
\usepackage{mathpazo}
\usepackage{bm}
\usepackage{tikz}
        \lstset{
    numbers = none ,                                    %可选参数有none,right,left
    breaklines ,                                        %换行有影响,不加这个则换行时从头开始
    numberstyle = \tiny ,                               %数字大小’
    keywordstyle = \color{blue!70} ,                    %关键字颜色
    commentstyle =\color{black!40!white} ,              %注释颜色
    frame = shadowbox ,                                 %阴影设置
    rulesepcolor = \color{red!20!green!20!blue!20} ,    %阴影颜色设置
    escapeinside =`',                                   %lst中文支持不太好,可以用这个括在中文旁边
    basicstyle =\footnotesize\ttfamily                  %代码字体设置
}
%NOTE THAT \usepackage{enumitem} \usepackage{bbding} is required
%basic defination
\newcommand{\introductionDefalutName}{内容提要}    %default title
\newcommand{\citshapeIntro}{\kaishu}            %font
\newcommand{\introDot}{\upshape\scriptsize\SquareShadowBottomRight}     %dots before items in introductions
\definecolor{structurecolor}{HTML}{0072E3} %box color
%settings of introduction environment
\tcbset{
    introductionsty/.style={
    enhanced,
    breakable,
    colback=structurecolor!10,
    colframe=structurecolor,
    fonttitle=\bfseries,
    colbacktitle=structurecolor,
    fontupper=\citshapeIntro,
    attach boxed title to top center={yshift=-3mm,yshifttext=-1mm},
    boxrule=0pt,
    toprule=0.5pt,
    bottomrule=0.5pt,
    top=8pt,
    before skip=8pt,
    sharp corners
    },
}

\newenvironment{introduction}[1][\introductionDefalutName]{
    \begin{tcolorbox}[introductionsty,title={#1}]
        \begin{multicols}{2}
            \begin{itemize}[label=\textcolor{structurecolor}{\introDot}]
}
{
            \end{itemize}
        \end{multicols}
    \end{tcolorbox}
}
%自定义盒子环境
%summarybox[color]{title}
\newenvironment{mybox}[2][\coldefault]{
    \begin{tcolorbox}[
        title=#2,
        colframe=#1,
        colback=#1!10,
        breakable
    ]
}{\end{tcolorbox}}
% \input{./mytemplate/components/myColorPlate.tex}  %provide colors, you can use the VsCode extension color highlight. e.g.#02a29f
%mydef{text}
\newcommand{\mydef}[1]{\textbf{\textcolor{blue!75!black}{#1}}}
\newcommand{\myemph}[1]{\textbf{\textcolor{red!85!black}{#1}}}

%myimage[figureScale]{path}{caption}{label}
\newcommand{\myimage}[4][1]{{
    \begin{figure}[htbp]
        \centering
        \label{fig:#4}
        \includegraphics[scale=#1]{#2}
        \caption{#3}
    \end{figure}}}
%basic defination
\newcommand{\citshapeRef}{\kaishu}            %font
\definecolor{colReference}{HTML}{FFA042}   %box color
%settings of introduction environment
\tcbset{
    referencesty/.style={
    enhanced,
    breakable,
    colback=colReference!10,
    colframe=colReference,
    fonttitle=\bfseries,
    colbacktitle=colReference,
    fontupper=\citshapeRef,
    leftrule=5pt,
    rightrule=0pt,
    toprule=0pt,
    bottomrule=0pt,
    top=8pt,
    before skip=8pt,
    sharp corners
    },
}
\newenvironment{myReference}{
    \begin{tcolorbox}[referencesty]
}
{
    \end{tcolorbox}
}
%自定义提示颜色
\definecolor{mycolwarning}{HTML}{CE0000}
\definecolor{mycolsummary}{HTML}{8CEA00}
\definecolor{mycoldefault}{HTML}{2828FF}
\definecolor{mycolquestion}{HTML}{00E3E3}
\definecolor{mycolexample}{HTML}{FFDC35}
\definecolor{mycolextend}{HTML}{FFA042}
% \newcommand{\mycolwarning}{__mycolwarning}
% \newcommand{\mycolsummary}{__mycolsummary}
% \newcommand{\mycoldefault}{__mycoldefault}
% \newcommand{\mycolquestion}{__mycolquestion}
% \newcommand{\mycolexample}{__mycolexample}
% \newcommand{\mycolextend}{__mycolextend}

% DENOTE THAT pifont package is required for command \ping

\definecolor{colRemark}{HTML}{73BF00}
\tcbset{
    remarksty/.style={
        colframe=magenta,
        colback=magenta!12!white,
        boxed title style={colback=magenta},
	    breakable,
	    enhanced,
	    sharp corners,
	    boxsep=1pt,
	    attach boxed title to top left={yshift=-\tcboxedtitleheight,  yshifttext=-.75\baselineskip},
	    boxed title style={boxsep=1pt,sharp corners},
        fonttitle=\bfseries\sffamily,
        drop lifted shadow
    },
}

\newtcolorbox{remark}[1][]{
    remarksty,
    title={\scalebox{1.75}{\raisebox{-.25ex}{\ding{43}}}~笔记},
    colframe=yellow!45!white,
    colback=yellow!45!white,
    coltitle=colRemark,
    fontupper=\sffamily,
    boxed title style={colback=yellow!45!white},
    boxed title style={boxsep=1ex,sharp corners},%%
    overlay unbroken and first={
        \node[below right,font=\normalsize,color=red,text width=.8\linewidth]
        at (title.north east) {#1};
    }
}
%Theorem environment
    %Here to define a counter
\newcounter{myTheoremCounter}[subsection]
    %Here to define a auto counter
\newcommand{\myAutoTheoremCounter}{\stepcounter{myTheoremCounter}\thesubsection.\themyTheoremCounter}
\definecolor{myTheoremColor}{HTML}{FFDC35}
\tcbset{
    myTheoremStyle/.style={
    enhanced,
    breakable,
    colback=myTheoremColor!10,
    colframe=myTheoremColor,
    fonttitle=\bfseries,
    colbacktitle=myTheoremColor,
    fontupper=\citshapeIntro,
    attach boxed title to top left={},
    boxed title style = {sharp corners},
    boxrule=1pt,
    toprule=1pt,
    bottomrule=1pt,
    top=8pt,
    before skip=8pt,
    sharp corners,
    },
}
%Title needed
\newenvironment{myTheorem}[1]{
    \begin{tcolorbox}[
        myTheoremStyle,
        title=\color{black}{定理\myAutoTheoremCounter\quad #1}
    ]
        
}{\end{tcolorbox}}
%Defination environment
    %Here to define a counter
    \newcounter{myDefinationCounter}[subsection]
    %Here to define a auto counter
\newcommand{\myAutoDefinationCounter}{\stepcounter{myDefinationCounter}\thesubsection.\themyDefinationCounter}
\definecolor{myDefinationColor}{HTML}{01814A}
\tcbset{
    myDefinationStyle/.style={
    enhanced,
    breakable,
    colback=myDefinationColor!10,
    colframe=myDefinationColor,
    fonttitle=\bfseries,
    colbacktitle=myDefinationColor,
    fontupper=\citshapeIntro,
    attach boxed title to top left={},
    boxed title style = {sharp corners},
    boxrule=1pt,
    toprule=1pt,
    bottomrule=1pt,
    top=8pt,
    before skip=8pt,
    sharp corners
    },
}
%Title needed
\newenvironment{myDefination}[1]{
    \begin{tcolorbox}[
        myDefinationStyle,
        title=\color{white}{定义\myAutoDefinationCounter\quad #1}
    ]
        
}{\end{tcolorbox}}
\newcounter{myPropertyCounter}
    %Here to define a auto counter
\newcommand{\myAutoPropertyCounter}{\stepcounter{myPropertyCounter}\themyPropertyCounter}
\definecolor{myPropertyColor}{HTML}{46A3FF}
\newenvironment{myProperties}{
    \setcounter{myPropertyCounter}{0}
    \begin{itemize}[label=\textbf{\color{myPropertyColor}{性质\myAutoPropertyCounter}}]

}{\end{itemize}}
\definecolor{__selWordColor__}{HTML}{EA7500} % default: 0,124,53
\definecolor{__ledLineColor__}{HTML}{EA7500} % default: 153,255,153
\definecolor{__sideBoxColor__}{HTML}{ACD6FF} % default: 216,255,216

\newcommand{\elegantpar}[2]{%
  \textcolor{__selWordColor__}{$\bm\langle{}\!{}$#1${}\!{}\bm\rangle$}%
  \begin{tikzpicture}[remember picture, baseline=-0.75ex]%
    \node[coordinate] (inText) {};%
  \end{tikzpicture}%
  \marginpar{%
    \renewcommand{\baselinestretch}{1.0}%
    \begin{tikzpicture}[remember picture]%
      \draw node[fill= __sideBoxColor__, rounded corners,text width=\marginparwidth] (inNote){\footnotesize#2};%
  \end{tikzpicture}%
  }%
  \begin{tikzpicture}[remember picture, overlay]%
    \draw[draw = __ledLineColor__, thick]
      ([yshift=-0.55em] inText)
        -| ([xshift=-0.55em] inNote.west)
        -| (inNote.west);%
  \end{tikzpicture}%
}

\setlength{\marginparwidth}{2.5cm}
    \end{lstlisting}

    其中myFavorPkgs用于导入依赖包,componentsIncluder用于导入定制模板。用户可以调整componentsIncluder中的内容
    进行模板的增删。\myemph{注意}myFavorPkgs.tex先导入。
    \newpage
    \section{模板介绍}
    以下模板定义于components文件夹下,可根据自己需要进行修改(如颜色,线条粗细等)
    \subsection{introduction}

    \begin{lstlisting}
\begin{introduction}
    \item   `极限'
    \item   `导数'
    \item   `数列收敛定理'
\end{introduction}
    \end{lstlisting}
    
    \begin{introduction}
        \item   极限
        \item   导数
        \item   数列收敛定理
    \end{introduction}
    
    \subsection{myBasicTcbcolorbox}
    

    \begin{lstlisting}
\begin{mybox}[purple]{`标题'}

\end{mybox}
    \end{lstlisting}

    注意第一个参数是可选参数,但是似乎把定义删去了(默认值是默认颜色),\myemph{需要用definecolor自定义颜色或默认提供的颜色},否则颜色会出错。

    第二个参数是盒子标题,如果为空,则不含标题栏。
    
    优势在于可以自定义颜色
    \begin{mybox}[purple]{标题}

    \end{mybox}

    \subsection{myColorPlate}
    
    定义了一大堆颜色,结合VsCode中的Color Highlight插件可以看到颜色。

    \subsection{myDefinationBox}
    
    
    定义框,注意section和subsection要有,否则计数器可能不是你想的那样。(作者能力有限QWQ)

    
    \begin{lstlisting}
\begin{myDefination}{`勾股定理'}
    \begin{myDefination}{`有理数'}
        `可用'$\frac p q$`表示的数,其中p,q为整数且互质。'
    \end{myDefination}
\end{myDefination}
    \end{lstlisting}
    
    \begin{myDefination}{有理数}
        可用$\frac p q$表示的数,其中p,q为整数且互质。
    \end{myDefination}
    
    \subsection{myLstListingSetting}
    
    用于全局定义listing环境,作者主要用于代码环境,可自行修改。

    \subsection{myProperty}
    
    用于定义性质环境。

    
    \begin{lstlisting}
\begin{myProperties}
    \item `人是会行走的动物。'
    \item `人会思考。'
    \item `人会交流。'
\end{myProperties}
    \end{lstlisting}
    
    
    \begin{myProperties}
        \item 人是会行走的动物。
        \item 人会思考。
        \item 人会交流。
    \end{myProperties}
    
    \subsection{myQuickFontSty}
    
    \begin{lstlisting}
`自定义文字字体,如作者使用了'\mydef{`定义'}`字样和'\myemph{`强调'}`字样。'
    \end{lstlisting}
    
    自定义文字字体,如作者使用了\mydef{定义}字样和\myemph{强调}字样。

    \subsection{myQuickImage}
    自定义命令快捷方式插入图片,但是因为有snippet的存在,可以弃用。
    \begin{lstlisting}
\myimage{scale}{imagePath}{caption}{label}
    \end{lstlisting}
    
    \subsection{myReference}
    引用环境。
    
    \begin{lstlisting}
\begin{myReference}
    `夏至至长,夏至致短。'
\end{myReference}
    \end{lstlisting}
    
    \begin{myReference}
        夏至至长,夏至致短。
    \end{myReference}
    
    \subsection{mySignalColor}
    此模板自定义颜色,以便于自己使用。可自定义。

    \subsection{myTheoremBox}
    
    定理框,注意section和subsection要有,否则计数器可能不是你想的那样。颜色可进入对应模板自定义(作者能力有限QWQ)

    
    \begin{lstlisting}
\begin{myTheorem}{`勾股定理'}
    \begin{equation}
        \label{eq:eq1}
            a^2+b^2=c^2
    \end{equation}
\end{myTheorem}
    \end{lstlisting}
    
    \begin{myTheorem}{勾股定理}
        \begin{equation}
            \label{eq:eq1}
                a^2+b^2=c^2
        \end{equation}
    \end{myTheorem}

    \subsection{remarkNote}
    笔记环境。
    
    \begin{lstlisting}
\begin{remark}
    `作者是个傻逼。'
\end{remark}
    \end{lstlisting}
    
    \begin{remark}
        作者是个傻逼。
    \end{remark}
    
\end{document}