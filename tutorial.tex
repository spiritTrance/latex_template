\documentclass{ctexart}

\usepackage[dvipsnames, svgnames, x11names]{xcolor}
\usepackage[top=2cm,bottom=2cm,left=1cm,right=4cm]{geometry}
\usepackage[bookmarks=true,bookmarksopen=true,bookmarksnumbered=true]{hyperref}
\usepackage{listings}
\usepackage{graphicx}
\usepackage{tabularx}
\usepackage[most]{tcolorbox}
\usepackage{amsmath}
\usepackage{multicol}
\usepackage{pifont}
\usepackage{enumitem}
\usepackage{bbding}
\usepackage{colortbl}
\usepackage{placeins}
\usepackage[morefloats = 18]{morefloats}
\usepackage{mathpazo}
\usepackage{bm}
\usepackage{tikz}
\lstset{
    numbers = none ,                                    %可选参数有none,right,left
    breaklines ,                                        %换行有影响,不加这个则换行时从头开始
    numberstyle = \tiny ,                               %数字大小’
    keywordstyle = \color{blue!70} ,                    %关键字颜色
    commentstyle =\color{black!40!white} ,              %注释颜色
    frame = shadowbox ,                                 %阴影设置
    rulesepcolor = \color{red!20!green!20!blue!20} ,    %阴影颜色设置
    escapeinside =`',                                   %lst中文支持不太好,可以用这个括在中文旁边
    basicstyle =\footnotesize\ttfamily                  %代码字体设置
}
%NOTE THAT \usepackage{enumitem} \usepackage{bbding} is required
%basic defination
\newcommand{\introductionDefalutName}{内容提要}    %default title
\newcommand{\citshapeIntro}{\kaishu}            %font
\newcommand{\introDot}{\upshape\scriptsize\SquareShadowBottomRight}     %dots before items in introductions
\definecolor{structurecolor}{HTML}{0072E3} %box color
%settings of introduction environment
\tcbset{
    introductionsty/.style={
    enhanced,
    breakable,
    colback=structurecolor!10,
    colframe=structurecolor,
    fonttitle=\bfseries,
    colbacktitle=structurecolor,
    fontupper=\citshapeIntro,
    attach boxed title to top center={yshift=-3mm,yshifttext=-1mm},
    boxrule=0pt,
    toprule=0.5pt,
    bottomrule=0.5pt,
    top=8pt,
    before skip=8pt,
    sharp corners
    },
}

\newenvironment{introduction}[1][\introductionDefalutName]{
    \begin{tcolorbox}[introductionsty,title={#1}]
        \begin{multicols}{2}
            \begin{itemize}[label=\textcolor{structurecolor}{\introDot}]
}
{
            \end{itemize}
        \end{multicols}
    \end{tcolorbox}
}
%自定义盒子环境
%summarybox[color]{title}
\newenvironment{mybox}[2][\coldefault]{
    \begin{tcolorbox}[
        title=#2,
        colframe=#1,
        colback=#1!10,
        breakable
    ]
}{\end{tcolorbox}}
% 
% TODO: classify the color and define color using newcommand

\definecolor{white_1}{HTML}{272727}        %#272727
\definecolor{white_2}{HTML}{3C3C3C}        %#3C3C3C
\definecolor{white_3}{HTML}{4F4F4F}        %#4F4F4F
\definecolor{white_4}{HTML}{6C6C6C}        %#6C6C6C
\definecolor{white_5}{HTML}{7B7B7B}        %#7B7B7B
\definecolor{white_6}{HTML}{9D9D9D}        %#9D9D9D
\definecolor{white_7}{HTML}{ADADAD}        %#ADADAD
\definecolor{white_8}{HTML}{d0d0d0}        %#d0d0d0
\definecolor{white_9}{HTML}{F0F0F0}        %#F0F0F0
\definecolor{blue_1}{HTML}{000093}         %#000093
\definecolor{blue_2}{HTML}{0000E3}         %#0000E3
\definecolor{blue_3}{HTML}{2828FF}         %#2828FF
\definecolor{blue_4}{HTML}{4A4AFF}         %#4A4AFF
\definecolor{blue_5}{HTML}{6A6AFF}         %#6A6AFF
\definecolor{blue_6}{HTML}{9393FF}         %#9393FF
\definecolor{blue_7}{HTML}{B9B9FF}         %#B9B9FF
\definecolor{blue_8}{HTML}{CECEFF}         %#CECEFF
\definecolor{blue_9}{HTML}{ECECFF}         %#ECECFF
\definecolor{blue2_1}{HTML}{484891}        %#484891
\definecolor{blue2_2}{HTML}{5A5AAD}        %#5A5AAD
\definecolor{blue2_3}{HTML}{7373B9}        %#7373B9
\definecolor{blue2_4}{HTML}{9999CC}        %#9999CC
\definecolor{blue2_5}{HTML}{B8B8DC}        %#B8B8DC
\definecolor{blue2_6}{HTML}{C7C7E2}        %#C7C7E2
\definecolor{blue2_7}{HTML}{D8D8EB}        %#D8D8EB
\definecolor{blue2_8}{HTML}{E6E6F2}        %#E6E6F2
\definecolor{blue2_9}{HTML}{F3F3FA}        %#F3F3FA
\definecolor{green_1}{HTML}{467500}        %#467500
\definecolor{green_2}{HTML}{73BF00}        %#73BF00
\definecolor{green_3}{HTML}{8CEA00}        %#8CEA00
\definecolor{green_4}{HTML}{A8FF24}        %#A8FF24
\definecolor{green_5}{HTML}{B7FF4A}        %#B7FF4A
\definecolor{green_6}{HTML}{C2FF68}        %#C2FF68
\definecolor{green_7}{HTML}{DEFFAC}        %#DEFFAC
\definecolor{green_8}{HTML}{EFFFD7}        %#EFFFD7
\definecolor{green_9}{HTML}{F5FFE8}        %#F5FFE8
\definecolor{green2_1}{HTML}{006030}       %#006030
\definecolor{green2_2}{HTML}{01814A}       %#01814A
\definecolor{green2_3}{HTML}{019858}       %#019858
\definecolor{green2_4}{HTML}{01B468}       %#01B468
\definecolor{green2_5}{HTML}{02DF82}       %#02DF82
\definecolor{green2_6}{HTML}{1AFD9C}       %#1AFD9C
\definecolor{green2_7}{HTML}{4EFEB3}       %#4EFEB3
\definecolor{green2_8}{HTML}{96FED1}       %#96FED1
\definecolor{green2_9}{HTML}{C1FFE4}       %#C1FFE4
\definecolor{brown_1}{HTML}{613030}        %#613030
\definecolor{brown_2}{HTML}{804040}        %#804040
\definecolor{brown_3}{HTML}{984B4B}        %#984B4B
\definecolor{brown_4}{HTML}{AD5A5A}        %#AD5A5A
\definecolor{brown_5}{HTML}{C48888}        %#C48888
\definecolor{brown_6}{HTML}{D9B3B3}        %#D9B3B3
\definecolor{brown_7}{HTML}{E1C4C4}        %#E1C4C4
\definecolor{brown_8}{HTML}{EBD6D6}        %#EBD6D6
\definecolor{brown_9}{HTML}{F2E6E6}        %#F2E6E6
\definecolor{brown2_1}{HTML}{844200}       %#844200
\definecolor{brown2_2}{HTML}{9F5000}       %#9F5000
\definecolor{brown2_3}{HTML}{BB5E00}       %#BB5E00
\definecolor{brown2_4}{HTML}{EA7500}       %#EA7500
\definecolor{brown2_5}{HTML}{FF9224}       %#FF9224
\definecolor{brown2_6}{HTML}{FFA042}       %#FFA042
\definecolor{brown2_7}{HTML}{FFBB77}       %#FFBB77
\definecolor{brown2_8}{HTML}{FFD1A4}       %#FFD1A4
\definecolor{brown2_9}{HTML}{FFE4CA}       %#FFE4CA
\definecolor{red_1}{HTML}{2F0000}          %#2F0000
\definecolor{red_2}{HTML}{600000}          %#600000
\definecolor{red_3}{HTML}{750000}          %#750000
\definecolor{red_4}{HTML}{CE0000}          %#CE0000
\definecolor{red_5}{HTML}{EA0000}          %#EA0000
\definecolor{red_6}{HTML}{FF2D2D}          %#FF2D2D
\definecolor{red_7}{HTML}{ff7575}          %#ff7575
\definecolor{red_8}{HTML}{FFB5B5}          %#FFB5B5
\definecolor{red_9}{HTML}{FFD2D2}          %#FFD2D2
\definecolor{blue_1}{HTML}{000079}         %#000079
\definecolor{blue_2}{HTML}{004B97}         %#004B97
\definecolor{blue_3}{HTML}{0066CC}         %#0066CC
\definecolor{blue_4}{HTML}{0072E3}         %#0072E3
\definecolor{blue_5}{HTML}{0080FF}         %#0080FF
\definecolor{blue_6}{HTML}{46A3FF}         %#46A3FF
\definecolor{blue_7}{HTML}{84C1FF}         %#84C1FF
\definecolor{blue_8}{HTML}{ACD6FF}         %#ACD6FF
\definecolor{blue_9}{HTML}{D2E9FF}         %#D2E9FF
\definecolor{yellow_1}{HTML}{424200}       %#424200
\definecolor{yellow_2}{HTML}{5B5B00}       %#5B5B00
\definecolor{yellow_3}{HTML}{8C8C00}       %#8C8C00
\definecolor{yellow_4}{HTML}{C4C400}       %#C4C400
\definecolor{yellow_5}{HTML}{F9F900}       %#F9F900
\definecolor{yellow_6}{HTML}{FFFF6F}       %#FFFF6F
\definecolor{yellow_7}{HTML}{FFFFAA}       %#FFFFAA
\definecolor{yellow_8}{HTML}{FFFFCE}       %#FFFFCE
\definecolor{yellow_9}{HTML}{FFFFF4}       %#FFFFF4
\definecolor{yellow2_1}{HTML}{5B4B00}      %#5B4B00
\definecolor{yellow2_2}{HTML}{977C00}      %#977C00
\definecolor{yellow2_3}{HTML}{AE8F00}      %#AE8F00
\definecolor{yellow2_4}{HTML}{D9B300}      %#D9B300
\definecolor{yellow2_5}{HTML}{FFDC35}      %#FFDC35
\definecolor{yellow2_6}{HTML}{FFE66F}      %#FFE66F
\definecolor{yellow2_7}{HTML}{FFF0AC}      %#FFF0AC
\definecolor{yellow2_8}{HTML}{FFF4C1}      %#FFF4C1
\definecolor{yellow2_9}{HTML}{FFFCEC}      %#FFFCEC
\definecolor{olive_1}{HTML}{616130}        %#616130
\definecolor{olive_2}{HTML}{707038}        %#707038
\definecolor{olive_3}{HTML}{949449}        %#949449
\definecolor{olive_4}{HTML}{AFAF61}        %#AFAF61
\definecolor{olive_5}{HTML}{B9B973}        %#B9B973
\definecolor{olive_6}{HTML}{C2C287}        %#C2C287
\definecolor{olive_7}{HTML}{D6D6AD}        %#D6D6AD
\definecolor{olive_8}{HTML}{DEDEBE}        %#DEDEBE
\definecolor{olive_9}{HTML}{E8E8D0}        %#E8E8D0
\definecolor{pink_1}{HTML}{600030}         %#600030
\definecolor{pink_2}{HTML}{9F0050}         %#9F0050
\definecolor{pink_3}{HTML}{D9006C}         %#D9006C
\definecolor{pink_4}{HTML}{FF0080}         %#FF0080
\definecolor{pink_5}{HTML}{FF60AF}         %#FF60AF
\definecolor{pink_6}{HTML}{FF79BC}         %#FF79BC
\definecolor{pink_7}{HTML}{ffaad5}         %#ffaad5
\definecolor{pink_8}{HTML}{FFD9EC}         %#FFD9EC
\definecolor{pink_9}{HTML}{FFECF5}         %#FFECF5
\definecolor{cyan_1}{HTML}{003E3E}         %#003E3E
\definecolor{cyan_2}{HTML}{007979}         %#007979
\definecolor{cyan_3}{HTML}{00AEAE}         %#00AEAE
\definecolor{cyan_4}{HTML}{00E3E3}         %#00E3E3
\definecolor{cyan_5}{HTML}{4DFFFF}         %#4DFFFF
\definecolor{cyan_6}{HTML}{80FFFF}         %#80FFFF
\definecolor{cyan_7}{HTML}{BBFFFF}         %#BBFFFF
\definecolor{cyan_8}{HTML}{D9FFFF}         %#D9FFFF
\definecolor{cyan_9}{HTML}{ECFFFF}         %#ECFFFF
\definecolor{cyan2_1}{HTML}{336666}        %#336666
\definecolor{cyan2_2}{HTML}{408080}        %#408080
\definecolor{cyan2_3}{HTML}{4F9D9D}        %#4F9D9D
\definecolor{cyan2_4}{HTML}{6FB7B7}        %#6FB7B7
\definecolor{cyan2_5}{HTML}{95CACA}        %#95CACA
\definecolor{cyan2_6}{HTML}{A3D1D1}        %#A3D1D1
\definecolor{cyan2_7}{HTML}{B3D9D9}        %#B3D9D9
\definecolor{cyan2_8}{HTML}{C4E1E1}        %#C4E1E1
\definecolor{cyan2_9}{HTML}{D1E9E9}        %#D1E9E9
\definecolor{purple_1}{HTML}{460046}       %#460046
\definecolor{purple_2}{HTML}{750075}       %#750075
\definecolor{purple_3}{HTML}{930093}       %#930093
\definecolor{purple_4}{HTML}{D200D2}       %#D200D2
\definecolor{purple_5}{HTML}{FF44FF}       %#FF44FF
\definecolor{purple_6}{HTML}{FF8EFF}       %#FF8EFF
\definecolor{purple_7}{HTML}{FFBFFF}       %#FFBFFF
\definecolor{purple_8}{HTML}{FFD0FF}       %#FFD0FF
\definecolor{purple_9}{HTML}{FFE6FF}       %#FFE6FF

%color define from my ColorPlate
\newcommand{\mySkyblue}{blue_7}  %provide colors, you can use the VsCode extension color highlight. e.g.#02a29f
%mydef{text}
\newcommand{\mydef}[1]{\textbf{\textcolor{blue!75!black}{#1}}}
\newcommand{\myemph}[1]{\textbf{\textcolor{red!85!black}{#1}}}

%myimage[figureScale]{path}{caption}{label}
\newcommand{\myimage}[4][1]{{
    \begin{figure}[htbp]
        \centering
        \label{fig:#4}
        \includegraphics[scale=#1]{#2}
        \caption{#3}
    \end{figure}}}
%basic defination
\newcommand{\citshapeRef}{\kaishu}            %font
\definecolor{colReference}{HTML}{FFA042}   %box color
%settings of introduction environment
\tcbset{
    referencesty/.style={
    enhanced,
    breakable,
    colback=colReference!10,
    colframe=colReference,
    fonttitle=\bfseries,
    colbacktitle=colReference,
    fontupper=\citshapeRef,
    leftrule=5pt,
    rightrule=0pt,
    toprule=0pt,
    bottomrule=0pt,
    top=8pt,
    before skip=8pt,
    sharp corners
    },
}
\newenvironment{myReference}{
    \begin{tcolorbox}[referencesty]
}
{
    \end{tcolorbox}
}
%自定义提示颜色
\definecolor{mycolwarning}{HTML}{CE0000}
\definecolor{mycolsummary}{HTML}{8CEA00}
\definecolor{mycoldefault}{HTML}{2828FF}
\definecolor{mycolquestion}{HTML}{00E3E3}
\definecolor{mycolexample}{HTML}{FFDC35}
\definecolor{mycolextend}{HTML}{FFA042}
% \newcommand{\mycolwarning}{__mycolwarning}
% \newcommand{\mycolsummary}{__mycolsummary}
% \newcommand{\mycoldefault}{__mycoldefault}
% \newcommand{\mycolquestion}{__mycolquestion}
% \newcommand{\mycolexample}{__mycolexample}
% \newcommand{\mycolextend}{__mycolextend}

% DENOTE THAT pifont package is required for command \ping

\definecolor{colRemark}{HTML}{73BF00}
\tcbset{
    remarksty/.style={
        colframe=magenta,
        colback=magenta!12!white,
        boxed title style={colback=magenta},
	    breakable,
	    enhanced,
	    sharp corners,
	    boxsep=1pt,
	    attach boxed title to top left={yshift=-\tcboxedtitleheight,  yshifttext=-.75\baselineskip},
	    boxed title style={boxsep=1pt,sharp corners},
        fonttitle=\bfseries\sffamily,
        drop lifted shadow
    },
}

\newtcolorbox{remark}[1][]{
    remarksty,
    title={\scalebox{1.75}{\raisebox{-.25ex}{\ding{43}}}~笔记},
    colframe=yellow!45!white,
    colback=yellow!45!white,
    coltitle=colRemark,
    fontupper=\sffamily,
    boxed title style={colback=yellow!45!white},
    boxed title style={boxsep=1ex,sharp corners},%%
    overlay unbroken and first={
        \node[below right,font=\normalsize,color=red,text width=.8\linewidth]
        at (title.north east) {#1};
    }
}
%Theorem environment
    %Here to define a counter
\newcounter{myTheoremCounter}[subsection]
    %Here to define a auto counter
\newcommand{\myAutoTheoremCounter}{\stepcounter{myTheoremCounter}\thesubsection.\themyTheoremCounter}
\definecolor{myTheoremColor}{HTML}{FFDC35}
\tcbset{
    myTheoremStyle/.style={
    enhanced,
    breakable,
    colback=myTheoremColor!10,
    colframe=myTheoremColor,
    fonttitle=\bfseries,
    colbacktitle=myTheoremColor,
    fontupper=\citshapeIntro,
    attach boxed title to top left={},
    boxed title style = {sharp corners},
    boxrule=1pt,
    toprule=1pt,
    bottomrule=1pt,
    top=8pt,
    before skip=8pt,
    sharp corners,
    },
}
%Title needed
\newenvironment{myTheorem}[1]{
    \begin{tcolorbox}[
        myTheoremStyle,
        title=\color{black}{定理\myAutoTheoremCounter\quad #1}
    ]
        
}{\end{tcolorbox}}
%Defination environment
    %Here to define a counter
    \newcounter{myDefinationCounter}[subsection]
    %Here to define a auto counter
\newcommand{\myAutoDefinationCounter}{\stepcounter{myDefinationCounter}\thesubsection.\themyDefinationCounter}
\definecolor{myDefinationColor}{HTML}{01814A}
\tcbset{
    myDefinationStyle/.style={
    enhanced,
    breakable,
    colback=myDefinationColor!10,
    colframe=myDefinationColor,
    fonttitle=\bfseries,
    colbacktitle=myDefinationColor,
    fontupper=\citshapeIntro,
    attach boxed title to top left={},
    boxed title style = {sharp corners},
    boxrule=1pt,
    toprule=1pt,
    bottomrule=1pt,
    top=8pt,
    before skip=8pt,
    sharp corners
    },
}
%Title needed
\newenvironment{myDefination}[1]{
    \begin{tcolorbox}[
        myDefinationStyle,
        title=\color{white}{定义\myAutoDefinationCounter\quad #1}
    ]
        
}{\end{tcolorbox}}
\newcounter{myPropertyCounter}
    %Here to define a auto counter
\newcommand{\myAutoPropertyCounter}{\stepcounter{myPropertyCounter}\themyPropertyCounter}
\definecolor{__myPropertyColor__}{HTML}{46A3FF}
\newenvironment{myProperties}{
    \setcounter{myPropertyCounter}{0}
    \begin{itemize}[label=\textbf{\color{__myPropertyColor__}{性质\myAutoPropertyCounter}}]

}{\end{itemize}}

\begin{document}
    \tableofcontents
    \newpage

    \section{前言}
    部分模板参考overleaf上的项目(\mydef{bbe Book Template}和\mydef{ElegantBook Template}),这些模板都很不错www,推荐去体验一下。

    \section{如何使用文档}
    在导言区引入下面语句进行使用:
    \begin{lstlisting}
        \usepackage[dvipsnames, svgnames, x11names]{xcolor}
\usepackage[top=2cm,bottom=2cm,left=1cm,right=4cm]{geometry}
\usepackage[bookmarks=true,bookmarksopen=true,bookmarksnumbered=true]{hyperref}
\usepackage{listings}
\usepackage{graphicx}
\usepackage{tabularx}
\usepackage[most]{tcolorbox}
\usepackage{amsmath}
\usepackage{multicol}
\usepackage{pifont}
\usepackage{enumitem}
\usepackage{bbding}
\usepackage{colortbl}
\usepackage{placeins}
\usepackage[morefloats = 18]{morefloats}
\usepackage{mathpazo}
\usepackage{bm}
\usepackage{tikz}
        \lstset{
    numbers = none ,                                    %可选参数有none,right,left
    breaklines ,                                        %换行有影响,不加这个则换行时从头开始
    numberstyle = \tiny ,                               %数字大小’
    keywordstyle = \color{blue!70} ,                    %关键字颜色
    commentstyle =\color{black!40!white} ,              %注释颜色
    frame = shadowbox ,                                 %阴影设置
    rulesepcolor = \color{red!20!green!20!blue!20} ,    %阴影颜色设置
    escapeinside =`',                                   %lst中文支持不太好,可以用这个括在中文旁边
    basicstyle =\footnotesize\ttfamily                  %代码字体设置
}
%NOTE THAT \usepackage{enumitem} \usepackage{bbding} is required
%basic defination
\newcommand{\introductionDefalutName}{内容提要}    %default title
\newcommand{\citshapeIntro}{\kaishu}            %font
\newcommand{\introDot}{\upshape\scriptsize\SquareShadowBottomRight}     %dots before items in introductions
\definecolor{structurecolor}{HTML}{0072E3} %box color
%settings of introduction environment
\tcbset{
    introductionsty/.style={
    enhanced,
    breakable,
    colback=structurecolor!10,
    colframe=structurecolor,
    fonttitle=\bfseries,
    colbacktitle=structurecolor,
    fontupper=\citshapeIntro,
    attach boxed title to top center={yshift=-3mm,yshifttext=-1mm},
    boxrule=0pt,
    toprule=0.5pt,
    bottomrule=0.5pt,
    top=8pt,
    before skip=8pt,
    sharp corners
    },
}

\newenvironment{introduction}[1][\introductionDefalutName]{
    \begin{tcolorbox}[introductionsty,title={#1}]
        \begin{multicols}{2}
            \begin{itemize}[label=\textcolor{structurecolor}{\introDot}]
}
{
            \end{itemize}
        \end{multicols}
    \end{tcolorbox}
}
%自定义盒子环境
%summarybox[color]{title}
\newenvironment{mybox}[2][\coldefault]{
    \begin{tcolorbox}[
        title=#2,
        colframe=#1,
        colback=#1!10,
        breakable
    ]
}{\end{tcolorbox}}
% 
% TODO: classify the color and define color using newcommand

\definecolor{white_1}{HTML}{272727}        %#272727
\definecolor{white_2}{HTML}{3C3C3C}        %#3C3C3C
\definecolor{white_3}{HTML}{4F4F4F}        %#4F4F4F
\definecolor{white_4}{HTML}{6C6C6C}        %#6C6C6C
\definecolor{white_5}{HTML}{7B7B7B}        %#7B7B7B
\definecolor{white_6}{HTML}{9D9D9D}        %#9D9D9D
\definecolor{white_7}{HTML}{ADADAD}        %#ADADAD
\definecolor{white_8}{HTML}{d0d0d0}        %#d0d0d0
\definecolor{white_9}{HTML}{F0F0F0}        %#F0F0F0
\definecolor{blue_1}{HTML}{000093}         %#000093
\definecolor{blue_2}{HTML}{0000E3}         %#0000E3
\definecolor{blue_3}{HTML}{2828FF}         %#2828FF
\definecolor{blue_4}{HTML}{4A4AFF}         %#4A4AFF
\definecolor{blue_5}{HTML}{6A6AFF}         %#6A6AFF
\definecolor{blue_6}{HTML}{9393FF}         %#9393FF
\definecolor{blue_7}{HTML}{B9B9FF}         %#B9B9FF
\definecolor{blue_8}{HTML}{CECEFF}         %#CECEFF
\definecolor{blue_9}{HTML}{ECECFF}         %#ECECFF
\definecolor{blue2_1}{HTML}{484891}        %#484891
\definecolor{blue2_2}{HTML}{5A5AAD}        %#5A5AAD
\definecolor{blue2_3}{HTML}{7373B9}        %#7373B9
\definecolor{blue2_4}{HTML}{9999CC}        %#9999CC
\definecolor{blue2_5}{HTML}{B8B8DC}        %#B8B8DC
\definecolor{blue2_6}{HTML}{C7C7E2}        %#C7C7E2
\definecolor{blue2_7}{HTML}{D8D8EB}        %#D8D8EB
\definecolor{blue2_8}{HTML}{E6E6F2}        %#E6E6F2
\definecolor{blue2_9}{HTML}{F3F3FA}        %#F3F3FA
\definecolor{green_1}{HTML}{467500}        %#467500
\definecolor{green_2}{HTML}{73BF00}        %#73BF00
\definecolor{green_3}{HTML}{8CEA00}        %#8CEA00
\definecolor{green_4}{HTML}{A8FF24}        %#A8FF24
\definecolor{green_5}{HTML}{B7FF4A}        %#B7FF4A
\definecolor{green_6}{HTML}{C2FF68}        %#C2FF68
\definecolor{green_7}{HTML}{DEFFAC}        %#DEFFAC
\definecolor{green_8}{HTML}{EFFFD7}        %#EFFFD7
\definecolor{green_9}{HTML}{F5FFE8}        %#F5FFE8
\definecolor{green2_1}{HTML}{006030}       %#006030
\definecolor{green2_2}{HTML}{01814A}       %#01814A
\definecolor{green2_3}{HTML}{019858}       %#019858
\definecolor{green2_4}{HTML}{01B468}       %#01B468
\definecolor{green2_5}{HTML}{02DF82}       %#02DF82
\definecolor{green2_6}{HTML}{1AFD9C}       %#1AFD9C
\definecolor{green2_7}{HTML}{4EFEB3}       %#4EFEB3
\definecolor{green2_8}{HTML}{96FED1}       %#96FED1
\definecolor{green2_9}{HTML}{C1FFE4}       %#C1FFE4
\definecolor{brown_1}{HTML}{613030}        %#613030
\definecolor{brown_2}{HTML}{804040}        %#804040
\definecolor{brown_3}{HTML}{984B4B}        %#984B4B
\definecolor{brown_4}{HTML}{AD5A5A}        %#AD5A5A
\definecolor{brown_5}{HTML}{C48888}        %#C48888
\definecolor{brown_6}{HTML}{D9B3B3}        %#D9B3B3
\definecolor{brown_7}{HTML}{E1C4C4}        %#E1C4C4
\definecolor{brown_8}{HTML}{EBD6D6}        %#EBD6D6
\definecolor{brown_9}{HTML}{F2E6E6}        %#F2E6E6
\definecolor{brown2_1}{HTML}{844200}       %#844200
\definecolor{brown2_2}{HTML}{9F5000}       %#9F5000
\definecolor{brown2_3}{HTML}{BB5E00}       %#BB5E00
\definecolor{brown2_4}{HTML}{EA7500}       %#EA7500
\definecolor{brown2_5}{HTML}{FF9224}       %#FF9224
\definecolor{brown2_6}{HTML}{FFA042}       %#FFA042
\definecolor{brown2_7}{HTML}{FFBB77}       %#FFBB77
\definecolor{brown2_8}{HTML}{FFD1A4}       %#FFD1A4
\definecolor{brown2_9}{HTML}{FFE4CA}       %#FFE4CA
\definecolor{red_1}{HTML}{2F0000}          %#2F0000
\definecolor{red_2}{HTML}{600000}          %#600000
\definecolor{red_3}{HTML}{750000}          %#750000
\definecolor{red_4}{HTML}{CE0000}          %#CE0000
\definecolor{red_5}{HTML}{EA0000}          %#EA0000
\definecolor{red_6}{HTML}{FF2D2D}          %#FF2D2D
\definecolor{red_7}{HTML}{ff7575}          %#ff7575
\definecolor{red_8}{HTML}{FFB5B5}          %#FFB5B5
\definecolor{red_9}{HTML}{FFD2D2}          %#FFD2D2
\definecolor{blue_1}{HTML}{000079}         %#000079
\definecolor{blue_2}{HTML}{004B97}         %#004B97
\definecolor{blue_3}{HTML}{0066CC}         %#0066CC
\definecolor{blue_4}{HTML}{0072E3}         %#0072E3
\definecolor{blue_5}{HTML}{0080FF}         %#0080FF
\definecolor{blue_6}{HTML}{46A3FF}         %#46A3FF
\definecolor{blue_7}{HTML}{84C1FF}         %#84C1FF
\definecolor{blue_8}{HTML}{ACD6FF}         %#ACD6FF
\definecolor{blue_9}{HTML}{D2E9FF}         %#D2E9FF
\definecolor{yellow_1}{HTML}{424200}       %#424200
\definecolor{yellow_2}{HTML}{5B5B00}       %#5B5B00
\definecolor{yellow_3}{HTML}{8C8C00}       %#8C8C00
\definecolor{yellow_4}{HTML}{C4C400}       %#C4C400
\definecolor{yellow_5}{HTML}{F9F900}       %#F9F900
\definecolor{yellow_6}{HTML}{FFFF6F}       %#FFFF6F
\definecolor{yellow_7}{HTML}{FFFFAA}       %#FFFFAA
\definecolor{yellow_8}{HTML}{FFFFCE}       %#FFFFCE
\definecolor{yellow_9}{HTML}{FFFFF4}       %#FFFFF4
\definecolor{yellow2_1}{HTML}{5B4B00}      %#5B4B00
\definecolor{yellow2_2}{HTML}{977C00}      %#977C00
\definecolor{yellow2_3}{HTML}{AE8F00}      %#AE8F00
\definecolor{yellow2_4}{HTML}{D9B300}      %#D9B300
\definecolor{yellow2_5}{HTML}{FFDC35}      %#FFDC35
\definecolor{yellow2_6}{HTML}{FFE66F}      %#FFE66F
\definecolor{yellow2_7}{HTML}{FFF0AC}      %#FFF0AC
\definecolor{yellow2_8}{HTML}{FFF4C1}      %#FFF4C1
\definecolor{yellow2_9}{HTML}{FFFCEC}      %#FFFCEC
\definecolor{olive_1}{HTML}{616130}        %#616130
\definecolor{olive_2}{HTML}{707038}        %#707038
\definecolor{olive_3}{HTML}{949449}        %#949449
\definecolor{olive_4}{HTML}{AFAF61}        %#AFAF61
\definecolor{olive_5}{HTML}{B9B973}        %#B9B973
\definecolor{olive_6}{HTML}{C2C287}        %#C2C287
\definecolor{olive_7}{HTML}{D6D6AD}        %#D6D6AD
\definecolor{olive_8}{HTML}{DEDEBE}        %#DEDEBE
\definecolor{olive_9}{HTML}{E8E8D0}        %#E8E8D0
\definecolor{pink_1}{HTML}{600030}         %#600030
\definecolor{pink_2}{HTML}{9F0050}         %#9F0050
\definecolor{pink_3}{HTML}{D9006C}         %#D9006C
\definecolor{pink_4}{HTML}{FF0080}         %#FF0080
\definecolor{pink_5}{HTML}{FF60AF}         %#FF60AF
\definecolor{pink_6}{HTML}{FF79BC}         %#FF79BC
\definecolor{pink_7}{HTML}{ffaad5}         %#ffaad5
\definecolor{pink_8}{HTML}{FFD9EC}         %#FFD9EC
\definecolor{pink_9}{HTML}{FFECF5}         %#FFECF5
\definecolor{cyan_1}{HTML}{003E3E}         %#003E3E
\definecolor{cyan_2}{HTML}{007979}         %#007979
\definecolor{cyan_3}{HTML}{00AEAE}         %#00AEAE
\definecolor{cyan_4}{HTML}{00E3E3}         %#00E3E3
\definecolor{cyan_5}{HTML}{4DFFFF}         %#4DFFFF
\definecolor{cyan_6}{HTML}{80FFFF}         %#80FFFF
\definecolor{cyan_7}{HTML}{BBFFFF}         %#BBFFFF
\definecolor{cyan_8}{HTML}{D9FFFF}         %#D9FFFF
\definecolor{cyan_9}{HTML}{ECFFFF}         %#ECFFFF
\definecolor{cyan2_1}{HTML}{336666}        %#336666
\definecolor{cyan2_2}{HTML}{408080}        %#408080
\definecolor{cyan2_3}{HTML}{4F9D9D}        %#4F9D9D
\definecolor{cyan2_4}{HTML}{6FB7B7}        %#6FB7B7
\definecolor{cyan2_5}{HTML}{95CACA}        %#95CACA
\definecolor{cyan2_6}{HTML}{A3D1D1}        %#A3D1D1
\definecolor{cyan2_7}{HTML}{B3D9D9}        %#B3D9D9
\definecolor{cyan2_8}{HTML}{C4E1E1}        %#C4E1E1
\definecolor{cyan2_9}{HTML}{D1E9E9}        %#D1E9E9
\definecolor{purple_1}{HTML}{460046}       %#460046
\definecolor{purple_2}{HTML}{750075}       %#750075
\definecolor{purple_3}{HTML}{930093}       %#930093
\definecolor{purple_4}{HTML}{D200D2}       %#D200D2
\definecolor{purple_5}{HTML}{FF44FF}       %#FF44FF
\definecolor{purple_6}{HTML}{FF8EFF}       %#FF8EFF
\definecolor{purple_7}{HTML}{FFBFFF}       %#FFBFFF
\definecolor{purple_8}{HTML}{FFD0FF}       %#FFD0FF
\definecolor{purple_9}{HTML}{FFE6FF}       %#FFE6FF

%color define from my ColorPlate
\newcommand{\mySkyblue}{blue_7}  %provide colors, you can use the VsCode extension color highlight. e.g.#02a29f
%mydef{text}
\newcommand{\mydef}[1]{\textbf{\textcolor{blue!75!black}{#1}}}
\newcommand{\myemph}[1]{\textbf{\textcolor{red!85!black}{#1}}}

%myimage[figureScale]{path}{caption}{label}
\newcommand{\myimage}[4][1]{{
    \begin{figure}[htbp]
        \centering
        \label{fig:#4}
        \includegraphics[scale=#1]{#2}
        \caption{#3}
    \end{figure}}}
%basic defination
\newcommand{\citshapeRef}{\kaishu}            %font
\definecolor{colReference}{HTML}{FFA042}   %box color
%settings of introduction environment
\tcbset{
    referencesty/.style={
    enhanced,
    breakable,
    colback=colReference!10,
    colframe=colReference,
    fonttitle=\bfseries,
    colbacktitle=colReference,
    fontupper=\citshapeRef,
    leftrule=5pt,
    rightrule=0pt,
    toprule=0pt,
    bottomrule=0pt,
    top=8pt,
    before skip=8pt,
    sharp corners
    },
}
\newenvironment{myReference}{
    \begin{tcolorbox}[referencesty]
}
{
    \end{tcolorbox}
}
%自定义提示颜色
\definecolor{mycolwarning}{HTML}{CE0000}
\definecolor{mycolsummary}{HTML}{8CEA00}
\definecolor{mycoldefault}{HTML}{2828FF}
\definecolor{mycolquestion}{HTML}{00E3E3}
\definecolor{mycolexample}{HTML}{FFDC35}
\definecolor{mycolextend}{HTML}{FFA042}
% \newcommand{\mycolwarning}{__mycolwarning}
% \newcommand{\mycolsummary}{__mycolsummary}
% \newcommand{\mycoldefault}{__mycoldefault}
% \newcommand{\mycolquestion}{__mycolquestion}
% \newcommand{\mycolexample}{__mycolexample}
% \newcommand{\mycolextend}{__mycolextend}

% DENOTE THAT pifont package is required for command \ping

\definecolor{colRemark}{HTML}{73BF00}
\tcbset{
    remarksty/.style={
        colframe=magenta,
        colback=magenta!12!white,
        boxed title style={colback=magenta},
	    breakable,
	    enhanced,
	    sharp corners,
	    boxsep=1pt,
	    attach boxed title to top left={yshift=-\tcboxedtitleheight,  yshifttext=-.75\baselineskip},
	    boxed title style={boxsep=1pt,sharp corners},
        fonttitle=\bfseries\sffamily,
        drop lifted shadow
    },
}

\newtcolorbox{remark}[1][]{
    remarksty,
    title={\scalebox{1.75}{\raisebox{-.25ex}{\ding{43}}}~笔记},
    colframe=yellow!45!white,
    colback=yellow!45!white,
    coltitle=colRemark,
    fontupper=\sffamily,
    boxed title style={colback=yellow!45!white},
    boxed title style={boxsep=1ex,sharp corners},%%
    overlay unbroken and first={
        \node[below right,font=\normalsize,color=red,text width=.8\linewidth]
        at (title.north east) {#1};
    }
}
%Theorem environment
    %Here to define a counter
\newcounter{myTheoremCounter}[subsection]
    %Here to define a auto counter
\newcommand{\myAutoTheoremCounter}{\stepcounter{myTheoremCounter}\thesubsection.\themyTheoremCounter}
\definecolor{myTheoremColor}{HTML}{FFDC35}
\tcbset{
    myTheoremStyle/.style={
    enhanced,
    breakable,
    colback=myTheoremColor!10,
    colframe=myTheoremColor,
    fonttitle=\bfseries,
    colbacktitle=myTheoremColor,
    fontupper=\citshapeIntro,
    attach boxed title to top left={},
    boxed title style = {sharp corners},
    boxrule=1pt,
    toprule=1pt,
    bottomrule=1pt,
    top=8pt,
    before skip=8pt,
    sharp corners,
    },
}
%Title needed
\newenvironment{myTheorem}[1]{
    \begin{tcolorbox}[
        myTheoremStyle,
        title=\color{black}{定理\myAutoTheoremCounter\quad #1}
    ]
        
}{\end{tcolorbox}}
%Defination environment
    %Here to define a counter
    \newcounter{myDefinationCounter}[subsection]
    %Here to define a auto counter
\newcommand{\myAutoDefinationCounter}{\stepcounter{myDefinationCounter}\thesubsection.\themyDefinationCounter}
\definecolor{myDefinationColor}{HTML}{01814A}
\tcbset{
    myDefinationStyle/.style={
    enhanced,
    breakable,
    colback=myDefinationColor!10,
    colframe=myDefinationColor,
    fonttitle=\bfseries,
    colbacktitle=myDefinationColor,
    fontupper=\citshapeIntro,
    attach boxed title to top left={},
    boxed title style = {sharp corners},
    boxrule=1pt,
    toprule=1pt,
    bottomrule=1pt,
    top=8pt,
    before skip=8pt,
    sharp corners
    },
}
%Title needed
\newenvironment{myDefination}[1]{
    \begin{tcolorbox}[
        myDefinationStyle,
        title=\color{white}{定义\myAutoDefinationCounter\quad #1}
    ]
        
}{\end{tcolorbox}}
\newcounter{myPropertyCounter}
    %Here to define a auto counter
\newcommand{\myAutoPropertyCounter}{\stepcounter{myPropertyCounter}\themyPropertyCounter}
\definecolor{__myPropertyColor__}{HTML}{46A3FF}
\newenvironment{myProperties}{
    \setcounter{myPropertyCounter}{0}
    \begin{itemize}[label=\textbf{\color{__myPropertyColor__}{性质\myAutoPropertyCounter}}]

}{\end{itemize}}
    \end{lstlisting}

    其中myFavorPkgs用于导入依赖包,c o m p o n e n t s I n c l u d e r用于导入定制模板。用户可以调整componentsIncluder中的内容
    进行模板的增删。\myemph{注意}myFavorPkgs.tex先导入。
    \newpage
    \section{模板介绍}
    以下模板定义于components文件夹下,可根据自己需要进行修改(如颜色,线条粗细等)
    \subsection{introduction}

    \begin{lstlisting}
\begin{introduction}
    \item   `极限'
    \item   `导数'
    \item   `数列收敛定理'
\end{introduction}
    \end{lstlisting}
    
    \begin{introduction}
        \item   极限
        \item   导数
        \item   数列收敛定理
    \end{introduction}
    
    \subsection{myBasicTcbcolorbox}
    

    \begin{lstlisting}
\begin{mybox}[purple]{`标题'}

\end{mybox}
    \end{lstlisting}

    注意第一个参数是可选参数,但是似乎把定义删去了(默认值是默认颜色),\myemph{需要用definecolor自定义颜色或默认提供的颜色},否则颜色会出错。

    第二个参数是盒子标题,如果为空,则不含标题栏。
    
    优势在于可以自定义颜色
    \begin{mybox}[purple]{标题}

    \end{mybox}

    \subsection{myColorPlate}
    
    定义了一大堆颜色,结合VsCode中的Color Highlight插件可以看到颜色。

    \subsection{myDefinationBox}
    
    
    定义框,注意section和subsection要有,否则计数器可能不是你想的那样。(作者能力有限QWQ)

    
    \begin{lstlisting}
\begin{myDefination}{`勾股定理'}
    \begin{myDefination}{`有理数'}
        `可用'$\frac p q$`表示的数,其中p,q为整数且互质。'
    \end{myDefination}
\end{myDefination}
    \end{lstlisting}
    
    \begin{myDefination}{有理数}
        可用$\frac p q$表示的数,其中p,q为整数且互质。
    \end{myDefination}
    
    \subsection{myLstListingSetting}
    
    用于全局定义listing环境,作者主要用于代码环境,可自行修改。

    \subsection{myProperty}
    
    用于定义性质环境。

    
    \begin{lstlisting}
\begin{myProperties}
    \item `人是会行走的动物。'
    \item `人会思考。'
    \item `人会交流。'
\end{myProperties}
    \end{lstlisting}
    
    
    \begin{myProperties}
        \item 人是会行走的动物。
        \item 人会思考。
        \item 人会交流。
    \end{myProperties}
    
    \subsection{myQuickFontSty}
    
    \begin{lstlisting}
`自定义文字字体,如作者使用了'\mydef{`定义'}`字样和'\myemph{`强调'}`字样。'
    \end{lstlisting}
    
    自定义文字字体,如作者使用了\mydef{定义}字样和\myemph{强调}字样。

    \subsection{myQuickImage}
    自定义命令快捷方式插入图片,但是因为有snippet的存在,可以弃用。
    \begin{lstlisting}
\myimage{scale}{imagePath}{caption}{label}
    \end{lstlisting}
    
    \subsection{myReference}
    引用环境。
    
    \begin{lstlisting}
\begin{myReference}
    `夏至至长,夏至致短。'
\end{myReference}
    \end{lstlisting}
    
    \begin{myReference}
        夏至至长,夏至致短。
    \end{myReference}
    
    \subsection{mySignalColor}
    此模板自定义颜色,以便于自己使用。可自定义。

    \subsection{myTheoremBox}
    
    定理框,注意section和subsection要有,否则计数器可能不是你想的那样。颜色可进入对应模板自定义(作者能力有限QWQ)

    
    \begin{lstlisting}
\begin{myTheorem}{`勾股定理'}
    \begin{equation}
        \label{eq:eq1}
            a^2+b^2=c^2
    \end{equation}
\end{myTheorem}
    \end{lstlisting}
    
    \begin{myTheorem}{勾股定理}
        \begin{equation}
            \label{eq:eq1}
                a^2+b^2=c^2
        \end{equation}
    \end{myTheorem}

    \subsection{remarkNote}
    笔记环境。
    
    \begin{lstlisting}
\begin{remark}
    `作者是个傻逼。'
\end{remark}
    \end{lstlisting}
    
    \begin{remark}
        作者是个傻逼。
    \end{remark}
    
\subsection{sideBox}
注意合理调整页边距,侧注方框宽度为2.5cm,在sideBox模板下可以修改;在包里面的geometry可以修改页边距。
\begin{lstlisting}
    `我所说的这句话在旁边可以'\elegantpar`\{引用\}\{这就是旁注\}!'
\end{lstlisting}

我所说的这句话在旁边可以\elegantpar{引用}{这就是旁注}!


\end{document}